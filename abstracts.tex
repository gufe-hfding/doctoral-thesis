\begin{abstract}
	数据海量化增长、计算云端化、应用多样复杂化等新的变化为安 全和隐私带来了巨大挑战,深入理解隐私并实现动态隐私保护变得尤为重要,实现隐私保护与数据效用平衡存在挑战。基于非密码学的隐私研究领域主要有隐私定义与度量、隐私分析与推断以及隐私保护算法等三方面科学问题。这些问题的解决能 够有助于该领域完善其基础理论支撑,可在保证其实用性基础上提高隐私定义形式化 及度量、隐私泄露机理、隐私保护方案的科学性,为平衡隐私保护与数据效用提供解决路径。
	
	针对上述隐私领域的关键科学挑战,本文针对数据开放共享场景,对基于非密码学隐私领域展开隐私度量、隐私分析、隐私保护,以及隐私保护与数据效用平衡研究,以信息论和博弈论为工具,研究理性隐私保护模型及应用,重点通过Shannon信息论构建隐私度量的统一模型和量化方法,并以此为基础对独立序列型数据和关联序列型数据的属性隐私分别构建了隐私分析推断模型和隐私分析强度量化方法,设计了一种风险自适应访问控制模型以实现动态自适应的隐私保护,并结合扩展式博弈和演化博弈分别提出了不同的理性隐私风险访问控制模型,通过访问请求隐私风险函数和数据效用函数实现均衡,以实现数据开放共享隐私保护与数据访问效用间的平衡。具体贡献有:

	\begin{enumerate}
		\item 基于Shannon通信模型提出了一个隐私定义及量化、隐私分析强度、隐私保护强度等通用的隐私通信模型,对不含敌手的隐私保护、含敌手的隐私保护、多隐私保护源的隐私保护等情境提出了隐私度量模型,以满足对隐私信息、隐私保护强度和敌手隐私分析强度度量需求。对整个隐私保护模型提出了隐私保护强度和敌手攻击强度的量化方法,为隐私泄露量化提供了支撑。%;针对位置隐私保护的应用场景,通过所提出的隐私通信模型给出了具体的隐私量化模型,对基于位置服务隐私保护机制和隐私分析与推测能力进行了度量及分析。
		
		\item 针对序列型数据共享场景中的独立基因数据属性隐私提出了一种基于概率推断的隐私分析模型。该模型通过对个体基因序列属性值存在的相互关联关系进行分析,构建目标属性值推断的敌手模型。在提出的敌手模型基础上,分别提出了基于改进的隐马尔可夫模型和基于回归卷积神经网络模型的基因序列隐私分析方法。以隐私度量模型为基础,定义了序列型数据属性隐私和量化方法,并应用于量化属性隐私泄露和敌手获取隐私量。实验表明,提出的方法比现有基因序列属性隐私分析模型和算法更优,敌手对属性隐私的错误率、不确定度降低,敌手获得隐私信息量都比已有的工作更优。
		
		\item 针对家族成员的关联基因序列数据共享场景,构建了基因序列属性隐私概率推断模型。该模型构建了以家族谱系结构和置信传播模型为基础的属性隐私敌手模型,并在所定义的序列型数据属性隐私量化方法的基础上,分析了家族成员共享部分隐私基因数据对其他家庭成员基因序列属性隐私的影响。实验和对比表明,家族成员共享个人基因隐私数据会严重泄露其他家族成员的隐私,通过网络公开基因数据和家族成员共享基因数据可大规模获取家族其他成员的基因属性隐私。所提出的方法比现有工作的结果更优,推断属性隐私的精准率更高,敌手对基因属性隐私的不确定更低,获取的基因属性隐私信息量更多。
		
		\item 针对数据共享应用的动态隐私保护需求,在XACML上扩展提出了一种面向隐私保护的风险自适应访问控制模型。该模型在隐私保护访问控制敌手模型基础上,在标准XACML框架中新增了策略风险评估、会话控制和风险消减服务三个组件,增强了其他组件。在新增组件中,以Shannon信息熵为工具,提出了访问请求风险定义和量化方法,对访问控制请求风险和用户自身风险结合,提出了访问请求类型判别方法,并通过访问风险量化及基于信用卡模型的激励机制,动态自适应地约束用户访问行为。对比和分析表明,所提出的模型和方法较现有的工作更加动态化,且实现了隐私保护,易用性更好。
				
		\item 运用Shannon信息和博弈论,提出了基于扩展式博弈的理性隐私风险访问控制模型。该模型在定义了隐私风险和隐私侵犯访问的概念之后,提出了基于博弈论的隐私风险访问控制模型框架和工作流程。利用Shannon信息提出了量化访问请求和用户的隐私风险值计算方法,提出了多轮二人博弈来刻画面向隐私保护的风险访问控制中访问者与数据服务提供者的冲突与合作关系。分析表明,在基于隐私风险访问控制的每一轮博弈中都存在子博弈精炼Nash均衡,可通过限制侵犯隐私的访问请求实现隐私保护与访问数据效用间的平衡,该方法比已有的工作更有优势,需要更少的辅助信息,提供更多的风险适应性和隐私保护强度。
		
		\item 提出了一种基于演化博弈的理性隐私风险自适应访问控制模型。该模型包含了新的隐私风险量化模块和演化博弈决策模块,首先基于信息量对访问请求的数据集隐私信息量进行量化,构造了访问请求隐私风险函数和用户隐私风险函数;其次,基于演化博弈在有限理性假设下构建多参与者的访问控制演化博弈模型,利用复制动态方程分析了博弈过程中动态策略选择和演化稳定状态形成机理,提出了博弈演化稳定策略的选取方法。仿真实验和对比表明,提出的访问控制模型能够有效动态自适应地保护隐私信息,具有更好的隐私风险适应性,有限理性参与者的动态演化访问策略选取更加符合实际场景。
	\end{enumerate}
	
	\keywords{隐私度量,隐私推断分析,理性隐私保护,信息论,博弈论}
\end{abstract}


\begin{englishabstract}
	Great challenges of data security and privacy are arising along with data growing massively, computing clouding, and application complicating. It is especially important to understand privacy and implement dynamic privacy preserving. And there is still a huge challenge in achieving balance between privacy protection and data utility. The non-cryptographic-based privacy research fields mainly include three aspects, i.e. privacy definition and quantification, privacy analysis and inference, and privacy preserving mechanism. The solution of these issues can help the community to improve its basic theoretical foundation, and provide solid scientificity for privacy definition and measurement, privacy breach mechanism and privacy preserving, and then provide a
	route to balance privacy protection and data utility. 
	
	To address the mentioned critical scientific challenges, this work focuses on data opening and sharing scenarios, and non-cryptographic privacy domain. We mainly conduct research on privacy quantification, privacy analysis attack, privacy preserving, and the balance between privacy protection and data utility by using information theory and game theory. Several specific advances aiming to achieve rational privacy preserving and its application are suggested. After proposing a unified privacy quantification model based on information communication model,  attribute privacy inference attack models on independent sequence data and related sequence data are suggested respectively, and the breached privacy and strength of adversaries are quantified by our proposed  privacy quantification model. Further, a risk adaptive based access control(RaBAC) model for dynamic privacy preserving  is proposed, And additionally, two rational privacy RaBAC models are proposed by using extensive game and evolutionary game, respectively. During the rational privacy RaBAC models, functions for estimating privacy risk value of access request and utility of data are suggested, and thus the balance between privacy protection and accessed data utility is achiein data opening and sharing scenario. More specific contributions of this thesis are as follows.
	\begin{enumerate}
		\item 	
		A unified privacy communication model for measuring privacy definition and quantity, strength of privacy analysis attack, and strength of privacy preserving mechanism, is proposed by using Shannon information. 
		Several privacy quantification models of scenarios such as privacy preserving with/without adversary, privacy preserving with multi-privacy resources, are suggested for the measuring requirements of privacy definition, privacy analysis attack and privacy preserving mechanism. Furthermore, methods for quantifying the strength of  privacy analysis attack and privacy preserving mechanism are proposed, and these methods provide support to measure the quantity of privacy disclosure, the strength of  privacy analysis attack and privacy preserving mechanism.% At last, a specific privacy quantification model for the scenario of location privacy preserving is addressed by the proposed unified privacy communication model, and we measure and analyze the abilities of location-based service privacy preserving mechanism and privacy analysis attack.

		\item 	
		A privacy analysis attack model based on probability inference is proposed for the privacy of independent genetic data attributes in sequential data sharing scenarios. The model analyzes the interrelationship between the individual gene sequence attribute values and constructs the adversary model of the target attribute value inference. Based on the proposed adversary model, genome sequence privacy analysis attack methods are proposed based on an improved hidden Markov model and regression convolutional neural network model, respectively. Based on the privacy quantification model, attribute privacy and quantification methods of sequence data are defined, and these definitions are applied to quantify attribute privacy leaks and adversary acquisition. Experiments show that the proposed method is better than the existing genome sequence attribute privacy analysis model and algorithm. The error rate and uncertainty of the attribute privacy of the adversary are reduced, and the amount of private information obtained by the adversary is more than the existing work.
		
		\item 	
		An attribute privacy probability inference model is constructed for family members' associated gene sequence data sharing scenarios. This model constructs an attribute privacy adversary model based on family pedigree structure and belief propagation model. Based on the defined sequence data attribute privacy quantification method, we analyze the impact of individual's sequence attribute privacy breached by using his family members sharing part of the private gene data. Experiments and comparisons show that family members sharing personal genome privacy data can seriously reveal the privacy of other family members. By publishing genetic data on the Internet and shared genetic data by family members, the gene attribute privacy of other family members can be attacked on a large scale. The proposed method is better than the results of the existing work, and the accuracy of the inferred attribute privacy is higher, the adversary has less uncertainty about genome attribute privacy, and acquires more genome privacy information.
		
		\item 	
		Aiming at the dynamic privacy protection requirements of data sharing applications, a risk adaptive based access control model for privacy preserving is proposed based on XACML. After proposing the privacy preserving access control adversary model, three components, namely risk estimation, session control and risk mitigation services are added to the standard XACML framework, and other components are enhanced. In the new components, definition and quantification method of access request risk are proposed by using Shannon information entropy.  The access request type discriminating method is proposed by combing access control request risk  and the user's own risk. By using quantification of access request risk and  credit card incentives, the system dynamically and adaptively constrain user access behaviors. The comparison and analysis show that the proposed model and method are more dynamic than the existing work, and achieve privacy protection and better usability.
		
		\item 		
		A extensive game based rational privacy RaBAC model is proposed by employing Shannon information and game theory. After defining the concept of privacy risk and privacy violation access, this thesis proposes a framework and workflow for privacy risk access control model based on game theory. Calculation methods of access request's privacy risk and  the user's privacy risk are propsed by using Shannon information.
		 The conflict and cooperation relationship between the user and data service provider in the RaBAC of privacy protection is proposed by multi-stage two-player game. The analysis shows that there is a sub-game refining Nash equilibrium in stage game of the privacy RaBAC, which can balance the privacy protection and access data utility by limiting the privacy violation access request. This method benefits more than the existing work. It has the advantage of requiring less auxiliary information and providing more risk adaptability and privacy preserving.
		
		\item 
		A evolutionary game based rational RaBAC model for privacy preserving is proposed. The model includes a new privacy risk estimation module and an evolutionary game module. Firstly, based on the amount of information, the privacy information of the data set of the access request is quantified, and the access request privacy risk function and the user privacy risk function are constructed. Secondly, the multi-participant access control evolutionary game model is constructed under the assumption of bounded rationality by using evolutionary game theory. The dynamic mechanism selection and evolution stable state formation mechanism in the game process are analyzed by the replication dynamic equation. The selection method of game evolution stability strategy is proposed. Simulation experiments and comparisons show that the proposed access control model can effectively and adaptively preserving private information, and has better privacy risk adaptability. The dynamic evolution of access policy selection of bounded rational participants is more in line with the actual scenario.
	\end{enumerate}
	\englishkeywords{Privacy quantification, Privacy inference attack, Rational privacy preserving, Information theory, Game Theory}
\end{englishabstract}