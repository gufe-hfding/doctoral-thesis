\begin{abstract}
	互联网、移动互联网和物联网快速发展,以及5G技术的不断推进和商用推广,社交网络、位置服务、医疗健康、生物基因、工业控制等海量数据被主动或被动采集、传输、存储、流转、分析并应用。数据海量化增长、网络跨域泛在、计算云端化、应用多样复杂化等新的变化为安全和隐私带来了巨大挑战,大量的病毒、漏洞、攻击和数据关联分析,致使隐私严重泄漏,引发了人们极大的担忧。因此,深入的理解隐私并保护隐私变得尤为重要。基于非密码学的隐私研究领域主要有隐私定义与度量、隐私分析与推断、以及隐私保护算法等三方面科学问题。这些问题的解决能够有助于该领域完善其基础理论支撑,可在保证其实用性基础上提高隐私定义形式化及度量、隐私泄露机理、隐私保护方案的科学性。
	
	面对上述隐私领域的主要科学问题挑战,本文主要针对数据开放共享场景下的基于非密码学隐私研究,展开隐私度量、隐私分析、隐私保护及隐私保护与数据效用平衡方面研究,旨在能够深入探究隐私基础理论,提高对隐私泄露及隐私保护机理的理解,以提出能够动态、自适应地对包含大量隐私信息的数据集进行隐私保护,并实现隐私保护与数据效用间的平衡。具体进展与创新有:
	\begin{enumerate}
		\item 针对目前缺乏对隐私定义、隐私分析、隐私保护等统一的度量方法和框架的问题。基于Shannon信息论的通信模型框架提出了几种隐私保护信息通信模型,对不含敌手的隐私保护、含敌手的隐私保护、多隐私保护源的隐私保护等不同情境提出了相应的模型进行建模,以满足对隐私度量、隐私保护机制效果度量和敌手隐私分析能力度量等需求。提出了隐私保护方法的强度和敌手攻击能力的量化,为隐私泄露的量化提供了一种支撑,对整个隐私保护过程中的保护机制、敌手能力都提供了量化方法;针对位置隐私保护的应用场景,通过所提出的隐私信息通信模型给出了具体的信息通信模型,对基于位置服务隐私保护机制和隐私分析与推测能力进行了度量及分析。
		
		\item 为了对序列型数据属性隐私具有更深刻的理解,分析属性隐私泄露的机理。针对序列型属性共享场景中的基因数据属性隐私提出了一种基于概率推断的隐私分析模型。该模型通过对单条敏感数据记录属性值存在的相互关联关系进行分析,构建目标属性值推断的敌手模型。在提出的敌手模型基础上,分别提出了基于改进的隐马尔可夫模型和基于卷积神经网络模型的两种不同基因序列属性隐私分析方法。所提出的方法针对不存在亲属关系的群体型基因序列数据共享场景,在所提出的隐私度量模型基础上,定义了序列型数据属性隐私和量化方法,并应用于分析隐私属性泄露情况,量化了隐私泄露量和敌手获取隐私量。实验表明,提出的方法比现有基因序列属性隐私分析模型和算法更优,敌手对属性隐私的错误率、不确定度降低,敌手获得隐私信息量都比已有的工作更优。
		
		\item 进一步针对家族成员的序列型基因数据共享场景,构建了相互关联的基因序列数据属性隐私概率推断模型。该模型构建了以家族族谱和置信传播模型为基础的属性隐私敌手模型,并在所定义的序列型数据属性隐私量化方法的基础上,分析了家族成员共享部分隐私基因数据对其他家庭成员基因序列属性隐私的影响。试验和对比表明,家族成员共享个人基因隐私数据会严重影响泄露其他家族成员的隐私,通过网络公开基因数据和家族成员共享基因隐私数据可大量分析获取家族其他成员的基因属性隐私。所提出的方法比现有工作的结果更优,推断属性隐私的精准率更高,敌手对基因属性隐私的不确定更低,获取的基因属性隐私信息量更多。
		
		
		\item 针对以数据为中心的开放系统或数据共享应用场景的细粒度访问控制隐私保护问题,在XACML上扩展提出了一种基于风险的自适应访问控制模型,以动态化在访问控制过程中保护数据隐私,约束隐私侵犯行为,激励诚实访问行为。在提出面向隐私保护的风险访问控制敌手模型基础上,对标准XACML框架进行了扩展,新增了策略风险评估、会话控制和风险消减服务三个组件,增强了策略执行、策略访问和策略信息组件。在新增的组件中,以Shannon信息熵作为工具,在所提出的隐私度量模型基础上,提出了基于风险的隐私定义和量化方法,对用户的访问控制请求风险和用户自身的风险类型结合,提出了访问请求类型判别方法,并通过风险隐私量化及基于信誉的激励机制,动态自适应地约束用户访问行为。对比和分析表明,所提出的模型和方法较现有的工作更加动态化,且实现了隐私保护,易用性更好。
		
		\item 针对访问控制隐私保护机制中难以实现隐私保护与数据效用间的平衡问题,以及现有访问控制模型的不足。在所提出的风险访问控制模型基础上,进一步运用Shannon信息和博弈论,提出了二人博弈的理性风险访问控制模型。在定义了隐私风险和隐私侵犯访问的概念之后,提出了基于博弈论风险的访问控制模型框架和工作流程。此外,还进一步改利用Shannon信息的定义提出了量化访问请求和用户的隐私风险值计算公式,强化了访问控制请求对数据隐私的刻画;以提出的理性风险访问控制模型、访问请求隐私 风险和用户隐私风险为基础,提出了多轮二人博弈来刻画描述面向隐私保护的风险访问控制中访问者与数据服务提供者的”隐私保护-数据服务“冲突与合作关系。分析表明,在基于风险的访问控制的每一轮博弈中都存在子博弈精炼纳什均衡,可以通过限制侵犯隐私的访问请求来保护隐私,该方法比已有的工作更有优势,需要更少的辅助信息,提供更多的风险适应性和隐私保护能力。
		
		\item 在提出的风险自适应访问控制模型和两方理性隐私风险访问控制模型的基础上,进一步提出一种面向隐私保护的多参与者理性风险自适应访问控制模型。新提出的模型包含了新的隐私风险量化模块和演化博弈决策模块。该模型首先基于信息量对访问请求的数据集隐私信息量进行量化,构造了访问请求隐私风险函数和用户隐私风险函数;其次,基于演化博弈在有限理性假设下构建多参与者的访问控制演化博弈模型,利用复制动态方程分析了访问控制参与者的动态策略选择和演化稳定状态形成机理,提出了隐私风险访问控制博弈演化稳定策略的选取方法。仿真实验和对比表明,所提出的访问控制模型能够有效动态自适应地保护敏感信息资源系统中的隐私信息,具有更好的隐私风险适应性,有限理性参与者的动态演化访问策略选取更加符合实际场景。
	\end{enumerate}
	
	\keywords{隐私度量,隐私推断分析,理性隐私保护,信息论,博弈论}
\end{abstract}


%\begin{englishabstract}
%	In the private sensitive date centralized and opening information systems, such as social network and health-care information system, large-scale users access such system, and a fine-grained and self-adaptive access control model for privacy preserving is desperately needed. Existed risk based, rational access control models cannot meet the requirement of adaptable privacy preserving, the participants of are entirely rational, which is too strong to match the real scenarios. To this end, a rational multi-player risk-adaptive based access control model for privacy preserving is proposed. In this model, novel modules of privacy risk estimation and evolutionary game modeling are suggested; firstly, the privacy risk values of access request and requester are formulized by the private information quantity of the requested dataset, and by using Shannon information; secondly, a risk-adaptive based access control evolutionary game model is constructed by using evolutionary game under the supposing of bounded rational players, furthermore, dynamic strategies of participants and formation mechanism of evolutionarily stable states are analyzed by using replicator dynamics equation, and the method of choosing evolutionary stable strategy is proposed. Simulation and comparison results show that, the proposed model is effective to dynamically and adaptively preserve privacy in private information system, is more privacy risk adaptive, and dynamic evolutionary access strategies of the bounded rational participants are more suitable for practical scenarios.
%	
%	\englishkeywords{Privacy quantification, Privacy inference attack, Rational privacy preserving, Information theory, Game Theory}
%\end{englishabstract}