\chapter{绪论}
\label{chap:intro}

\section{研究背景及意义}
互联网、移动互联网和物联网快速发展,以及5G技术的不断推进和商用推广,社交网络、位置服务、医疗健康、生物基因、工业控制等海量数据被主动或被动采集、传输、存储、流转、分析并应用。海量数据的产生和应用推动了云计算、大数据和边缘计算等新兴产业和技术的爆发式增长,并产生了智慧医疗、智慧交通、智慧政府、智慧城市等不同的应用,极大地丰富了人们的物质和精神生活。同样,数据海量化增长、网络跨域泛在、计算云端化、应用多样复杂化等新的变化为安全和隐私带来了巨大挑战,大量的病毒、漏洞、攻击和数据关联分析,致使隐私严重泄漏,引发了人们极大的担忧。表\ref{tab:privacy_leakeges}展示了近年来主要的隐私泄露事件,充分表明了隐私泄露已经成为网络空间的重要威胁。在此背景下,深入的理解隐私并保护隐私变得尤为重要。

\begin{table}[htbp]
\caption{近年来主要隐私泄露事件简况}
\label{tab:privacy_leakeges}
\centering
\begin{tabular}{p{0.12\textwidth}p{0.22\textwidth}p{0.25\textwidth}p{0.25\textwidth}}%

	\toprule
	\textbf{时间}&\textbf{事件}&\textbf{影响}&\textbf{原因}\\
	\midrule
	2017年7月 & 韩国加密货币交易所客户数据泄露 & 3万个人用户数据被盗并遭受电话诈骗 & 黑客入侵攻击\\
	2017年10月 & 全球11个国家41个凯悦酒店数据泄露 & 数据量不详,涵盖信用卡姓名、卡号、到期日期、验证码等 & 通过恶意软件进行黑客入侵\\
	2017年10月 & 马来西亚超过总人口的手机用户信息泄露 &4620万人用户地址、身份证号、手机识别卡信息泄露 & 不详\\
	2017年10月 & 埃森哲服务器大量敏感信息泄露 & 19亿敏感的密码和解密密钥泄露 & 操作失误将数据放到未保护的云服务上\\
	2017年10月 & 南非史上最大规模数据泄露 & 3160万人个人资料被公之于众 & 数据在未保护的服务器上导致黑客窃取\\
	2018年3月 & Facebook用户数据泄露 & 5千万用户数据泄露,影响美国大选 & 越权采集并分析用户喜好、性格、行为特点、政治倾向\\
	2018年8月 & 华住集团数据泄露 & 5亿条、140G华住旗下酒店的用户数据泄露 & 不详\\
	2018年8月 & 谷歌采集设备、地图、搜索位置信息 & 全球超20亿用户数据被越权采集 & 谷歌公司故意采集\\
\bottomrule
\end{tabular}
\end{table}

由于90\%以上的数据被提供公共服务的政府、社会组织和企业所采集、存储,为了使数据发挥更大的价值,往往需要对包含大量隐私信息的数据进行共享、开放、交换和分析处理;同时很多信息服务也是基于个人隐私信息与服务质量的交换,如网站注册服务、公共WIFI接入、云存储、智能手机导航、信息搜索与广告推送、在线信用卡支付、RFID应用等。这些场景中由于法律法规要求和个人意愿,需要对隐私信息进行保护,同时服务提供方、数据利用方或恶意第三方希望获取更多的隐私敏感信息,以提供更好的服务、获取更大数据价值,得到更好的数据效用,两个目标同时存在且相互冲突,需要均衡解决。

关于隐私的研究,自2006年~$k$~匿名模型\cite{sweeney2002k}被提出以后逐步变成系统化的研究,隐私研究发展为基于密码学的方案\cite{nabeel2014privacy,huang2015review}和基于非密码学的方案\cite{sweeney2002k,machanavajjhala2007l,li2007t,dwork2006differential,zhang2018privacy}两大类,这些方案被大规模应用于以数据为中心的开放、复杂、跨域场景中,如云存储、社交网络、基于位置服务、物联网、边缘计算、数据挖掘、机器学习、医疗健康等。众多应用场景中,隐私保护目标和数据利用目标天然矛盾,如何平衡二者的关系是核心问题之一。在这两类隐私研究中,基于密码学的方案通常利用可证明安全理论定义密码学意义上的隐私保护目标,设计对应的密码学方案,如同态加密、可搜索加密、属性密码方案等实现隐私保护目标\cite{nabeel2014privacy,huang2015review};基于非密码学的方案主要是定义了匿名性设计达到匿名化效果的算法来实现用户的身份匿名隐私保护\cite{sweeney2002k,machanavajjhala2007l,li2007t},通过定义邻近数据集的查询结果不可区分性,设计加噪的方法达到这种不可区分性来实现属性值的隐私保护\cite{dwork2006differential},通过定义数据动态隐私,设计自适应的风险的细粒度访问控制实现隐私数据不被非授权用户访问\cite{zhang2018privacy}。其中,基于密码学的方案具有严格的理论方法支撑,能够达到预期的隐私保护目标,但是这些隐私定义是密码学意义上安全性定义,隐私保护方案设计也依赖公钥密码,其计算高度复杂导致效率低下,且难以采用折中的措施实现隐私保护效果和数据效用的平衡;基于非密码学的方案通过概率或信息论定义匿名性和不可区分性意义上的隐私,并设计泛化匿名或加噪的方式实现匿名或属性值隐私保护,效率高且有利于平衡隐私保护效果和数据效用。目前,以数据为中心的开放应用场景多样化,特别是数据开放共享应用中,大规模的个人隐私需要在保证数据可用的前提下得到实用性的隐私保护,研究基于非密码学的方案可以达到这一目标,平衡隐私保护与数据效用,具有重要的现实意义。

隐私领域的研究主要有三方面科学问题。\textbf{第一、隐私定义与度量}。如何恰当形式化的定义隐私、并对隐私进行量化。特别是隐私量化,既包括对特定数据集中隐私量的量化,又包括在某种隐私分析攻击模型下,个人隐私潜在泄露量、隐私分析攻击后隐私泄露量评估,还包括某一隐私保护模型对数据集隐私保护能力的量化。\textbf{第二、隐私分析与推测}。在某一场景下针对保护后的隐私信息数据集进行隐私分析与推测,如何最大程度的获取更多隐私信息。\textbf{第三、隐私保护}。如何对某一场景下的隐私数据集进行有效隐私保护,如何在保护隐私的同时平衡隐私保护效果和数据效用。深入研究科学问题一和科学问题二有助于对隐私的理解和认识,能够对隐私泄露的机理进行深入剖析,能够对设计更好的隐私保护方案提供科学理论依据和评价方法,研究科学问题三能够实现对数据隐私的预期性保护,如可量化的、动态性的、自适应的隐私保护,能够平衡隐私保护效果与数据效用间的关系。上述三个科学问题对基于非密码学的方案研究有重要的理论意义,能够有助于该领域完善其基础理论支撑,可在保证其实用性基础上提高隐私定义形式化及度量、隐私泄露机理、隐私保护方案的科学性。

\textit{本文主要针对数据开放共享场景下的基于非密码学隐私研究领域,展开隐私定义与度量、隐私分析与推测及隐私保护研究,旨在提出能够动态、自适应地对包含大量隐私信息的数据集进行隐私保护,并实现隐私保护与数据效用间的平衡。}

\section{研究现状}

本节围绕本文的研究内容,就相关研究领域的现状进行梳理和分析,包括隐私度量、隐私分析与推测以及隐私保护三个方面,以更加深入的理解本文研究的背景。

\subsection{隐私度量}
早期对隐私的认知是法理上的“隐私权”,在技术上被定义为匿名性,即在一个匿名集中元素不能被唯一标识的状态。在匿名通信系统中,匿名性最初被量化为匿名数据集阶的自然对数~$A=log_2(N)$\cite{reiter1998crowds},未考虑敌手获取的概率信息量。直到2002年,匿名性的量化引入了信息论\cite{serjantov2002towards}刻画敌手对匿名系统或匿名集的去匿名化攻击后获得的信息量,Serjantov 和Danezis\cite{serjantov2002towards}将匿名性定义为~$d=H(X)$,其中$H(X)$是攻击者对匿名集合中的元素进行去匿名分析后的信息熵;随后,利用正规熵\cite{diaz2002towards}、相对熵\cite{deng2006measuring}有不同的匿名性度量方法被提出,这些匿名性度量方法都未考虑敌手的背景知识,无法动态量化匿名性。2007年,Edman等\cite{edman2007combinatorial}利用二分图邻接矩阵和信息熵对传输信息者和接收信息者的信息进行映射,量化匿名性的信息量。这些匿名性的量化仅针对匿名通信系统(如洋葱路由系统,Tor系统和Crowds系统),更多此类方法见2009年Edman和Yener的综述\cite{edman2009anonymity},但这些方法并不适用数据共享和应用中的匿名性度量。针对数据共享应用的隐私量化最早聚焦在数据库领域,2002年,Sweeney\cite{sweeney2002k}将数据集中某一记录的匿名性量化为$d=1/k$,其中~$k$~是数据集中与该记录不可区分的记录数量;随后,该方法被扩展为~$l$~多样性匿名\cite{machanavajjhala2007l}和~$t$~邻近匿名\cite{li2007t}。针对数据集的匿名性定义被扩展到了基于位置服务\cite{niu2014achieving}、社交网络\cite{campan2008data}等应用场景,并用以不同形式的数据发布\cite{wong2006k,ying2009comparisons}。这些方法都是将匿名性量化为某一概率值,并不能对敌手去匿名化攻击获取的信息量进行量化,且无法根据敌手的背景知识进行动态量化。

林欣等\cite{lin2009lbs}对位置$k$匿名无法在连续查询攻击下刻画匿名集中位置的匿名度,提出了匿名集查询结果信息熵的匿名度量化方法$AD(q)=2^{H(q)}$,具有更好的适用性.

\subsection{隐私攻击与推测}

对基于位置服务中用户的位置信息进行直接~$k$~匿名保护的情况,林欣等提出了一种连续查询攻击\cite{lin2009lbs},在不同~$k$~匿名保护算法下的位置查询中成功区别出位置发送者。

\subsection{隐私保护算法}

鉴于基于密码学的隐私研究并非本文研究的聚焦点,尽管该领域亦有很多成果,本文也不再进行详述,可查阅基于属性密码\cite{edemacu2019privacy}、可搜索加密\cite{boesch2014survey,poh2017searchable}、同态密码\cite{acar2018survey}、安全多方计算\cite{cramer2015secure,dugan2016survey}等领域的综述进一步了解。


\section{有待解决的关键问题}


\section{本文工作}

\section{论文结构}