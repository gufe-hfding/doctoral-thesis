\chapter{基于信息通信模型的隐私度量模型}
\label{chap:entropy-metric-model}

\section{引言}

隐私保护的研究起步较早,但近年来突然受到产业界和学术界的广泛关注是因为大数据的不期而至.坦率地说,大数据的迅速发展让学术界始料未及,大数据的理论研究已经落后于产业需求,尤其是隐私保护成为大数据应用的主要瓶颈,移动网络、社交网络、基于位置服务等新型应用服务的推进,隐私问题更加突出.目前关于隐私保护有两个方向值得关注:一是研究隐私保护算法以更加有效的方式保护隐私;二是通过研究隐私泄露风险分析与评估,解决数据的可用性与隐私保护之间的的平衡.隐私保护算法目前主要集中在匿名方法,包括~$k$~匿名、~$l$~多样性匿名和~$t$~接近匿名及其衍生的方法.隐私度量最早起源于相关匿名算法[1], 在匿名隐私保护算法的研究过程中,不时有学者关注隐私量化问题,尤其是在定位服务领域,位置匿名及轨迹匿名算法上已有不少隐私度量的相关研究[2,3],因此对于隐私保护算法来说,隐私度量仍需进一步深入研究.然而就目前来说,隐私泄露涉及因素众多,设计有效的隐私保护算法仍然是挑战性问题,但政府及企业数据开放共享中迫切的隐私保护需求,促使我们不得不在可用性与隐私泄露之间寻求一种平衡,要解决这个问题,隐私风险分析及评估不失为一种方法.风险分析依然涉及到隐私量化问题,也就是说量化风险评估不失为隐私保护一种可行的解决方案,量化隐私风险必然也涉及隐私度量问题.从这些分析来看,隐私度量的研究具有十分重要的理论意义和应用价值.

信息熵作为信息度量的有效工具,在通信领域已展现出其重要的贡献[4].隐私作为一种信息,自然可以考虑用熵来量化,为此,不少学者或多或少进行了探索,比如事件熵、匿名集合熵、条件熵等[5-7],但其研究还较为零散,更多是针对某一具体领域,如位置隐私保护领域,目前尚未形成统一的模型及体系,其应用范围也受到限制,特别是隐私是具有时空性的,与人的主观感受也有关系,不同的人对同一隐私的认同可能不同. 鉴于以上分析,本文旨在参考Shannon信息论的通信框架[8],提出几种隐私保护信息熵模型,包括隐私保护基本信息熵模型、含敌手攻击的隐私保护信息熵模型、带主观感受的信息熵模型和多隐私信源的隐私保护信息熵模型.在这些模型中,将信息拥有者假设为发送方,隐私谋取者假设为接收方,隐私的泄露渠道假设为通信信道;基于这样的假设,分别引入信息熵、平均互信息量、条件熵及条件互信息等来分别描述隐私保护系统信息源的隐私度量、隐私泄露度量、含背景知识的隐私度量及泄露度量;以此为基础,进一步提出了隐私保护方法的强度和敌手攻击能力的量化测评,力图为隐私泄露的量化风险评估提供一种理论支持.

